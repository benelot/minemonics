\documentclass[main]{subfiles}

\begin{document}
% Chapter Template
\setcounter{chapter}{0}

%  Locomotion
%  A more natural approach
%   Evolutionary Algorithms
%   Chaotic systems as a source of high variability
%   Simple limiter control
\chapter{Introduction} % Main chapter title

\label{Chapter\thechapter} % Change X to a consecutive number; for referencing this chapter elsewhere, use \ref{ChapterX}

\lhead{Chapter \thechapter. \emph{Introduction}} % Change X to a consecutive number; this is for the header on each page - perhaps a shortened title

\section{Relating adaptability and locomotion}
% rev. 2

Motion is a central aspect of life for organisms on earth. In order to survive and be competitive in the endless development of evolution, one has to be highly adaptive on all levels from the microscopic level of the DNA to the macroscopic level of the morphology of an organism. The adaptability of an organism permits it to find a niche, live in it and adapt to it, in order to simplify the exploitation of the niche's environment. Except for primary producers that have found a way to be adaptable completely without self-generated motion and mainly relying on environmental forces such as wind or water to distribute seeds, organisms of higher complexity such as vertebrates, without exception, depend on locomotion, usually the most complex motor behavior the animal perform \cite{bib:Grillner2000}. Since higher complexity in general needs more space and therefore these organisms are larger than those of lower complexity, it does not suffice to rely on environmental forces to relocate. The ability to locomote enables an organism to extend the size of its current niche or to move to another niche if the attractivity of its current niche is reduced due a strong competitor or due to environmental changes. Locomotion is therefore a macroscopic key component to adaptive organisms. However, depending on the medium locomotion takes place in, it is hard to keep it stable. In water, fluid streams can prevent an organism from moving into one direction stably. On land, an organism faces problems with different obstacles and appropriate foot placement or switching between different gait patterns. How nature solves these problems is largely unknown, recent papers mainly focus on computationally expensive engineering solutions into which we will look in the next section.

\section{The engineering approach on locomotion}


Recent work shows experimental evidence that fishes for instance exploit fluidal vortices to reduce locomotory costs ( Müller \cite{bib:Muller2003}, Liao et al. \cite{bib:Liao2003a,bib:Liao2003b}).

\todo[inline]{Describe some engineering approaches on how to build robots that locomote.}

\section{A more natural approach}
% rev. 1

Nature seems to have solved the locomotion problem of getting from point A to point B extremely easily and elegantly. Even under constraints of rough terrain and obstacles interfering with the goal, a solution is nearly always found and does generally not include a long path planning or foot placement time. It seems as if the locomotion movements arise naturally from the interaction of the legs and the ground, because we rarely see an animal looking at its feet while it runs. Therefore the locomotion must be very robust to disturbances. On the other hand, engineering solutions usually rely on longer planning times and still do not reach competitive solutions in terms of leg placement and adaptability to the environment. Conventionally, the need for adaption has to be detected and then a new gait or leg position has to be calculated. Using a system that adaptively produces a variety of periodic movements, the calculation intensive controller could be replaced by the periodic controller. The interaction of the body and the controller with the environment automatically constrains the motion space of the legs in order to move the leg appropriately and make the creature move forward. An example of periodicity generating controller is a chaotic, dynamical system that is controlled using simple limiter control. The sensors from the joints as well as the weight, shape and inertia of the limbs act as limiters. The goal of the thesis is to use evolutionary algorithms and an example of genotype to phenotype transcription to let creatures evolve in a rigid-body engine to show examples of evolved creatures that use the above mentioned controlled, chaotic system to move on different kinds of ground and adapt its movement depending on the ground.

\subsection{Evolutionary Algorithms}
% rev. 1

Evolutionary algorithms mimic the general way of how evolution as a self-optimizing process finds competitive individuals able to survive on earth. The general approach is based on a population of individuals, each individual based on a genotype, an element defining the characteristics of the individual. The genotype is the blueprint of the animal and subject to variation through mutation and crossover. The fitness functions, which model the environmental constraints that define what has to be achieved in order to be called fit, are applied to all individuals of the population to measure their fitness. If an individual is competitive with respect to the fitness functions, it has a higher chance to reproduce than a non-competitive individual. This means that the genotypes of individuals with high fitness values then are crossed, which forms a new genotype which is a mixture of the parent's genotypes. All individuals are subject to one or multiple types of mutations, which lead to slight changes in the genotype, thereby also leading to changes in the fitness of that individual. In order to keep the size of the population constant, some of the worst performing individuals are culled. Using this process, approximately optimal solutions to NP-complete problems can be found such as the travelling salesman problem [?look at Ruedi's notes]. This is due to the features of the variation operators, crossover leading to mixtures of different solutions which possibly are better solutions than the crossed solutions, and mutation helping to avoid getting stuck in local minima. The evolutionary algorithm therefore follows the scheme of a general gradient descent algorithm. It must be noted that the process per se does not aim to generate better individuals and that variational operations do not necessarily lead to a better fitness of an individual, but the population under the evolutionary process will adapt to the fitness landscape meaning that invidual solutions will converge to global minima. Very fit individuals will arise from it, not necessarily showing the same or even similar solution depending on the complexity of the fitness landscape (defining the number of peaks in the fitness landscape). 

The locomotion task could be a challenging optimization task for evolution due to the very non-continuous the fitness landscape. The lack of symmetry for instance, constituting a simple change in some genome structures, can instantly make the locomotion performance drop, as the creature loses a symmetric limb, making it much harder to walk properly due to the imbalance. Therefore the gradient on the fitness landscape might be of little help in some areas, degrading the algorithm to random sampling. Additionally, the representation of a locomotion solution might require a large parameter space, which slows down the convergence to good solutions. It will have to be seen how evolution will deal with these problems.

\subsection{Chaotic systems as a source of high variability}
% rev. 1

Chaotic systems are the underlying model of many different physical processes such as the three-body problem in astronomy studied by Poincaré \cite{bib:Poincare1892}, population models in biology studied by Verhulst \cite{bib:Verhulst1838}, weather models in meteorology studied by Lorenz \cite{bib:Lorenz1963} and many other models from different fields. Before the beginning of chaos theory dated back to the 19th century, it was thought that approximate knowledge about the initial state of a deterministic system was sufficient to approximate the knowledge about the future. In natural sciences, the exact knowledge of a real-world state of a system is impossible, as the real-valued nature of the natural, continuous system state has infinite precision. However an experiment can only reveal a limited-precision state of the system. Chaos theory showed that a small variation of the initial conditions causes a large variation in the final prediction, as already Poincaré put it when studying the three-body problem. The former belief that deterministic systems can be perfectly predicted turned out wrong for most systems \cite{bib:Motter2013}. This property leading to the unpredictability, namely the sensitivity to the initial conditions, can be used to produce a high variability of trajectories. In a chaotic system, the trajectories are guided by an infinite amount of unstable periodic orbits(UPOs), that in interaction with each other lead to the variability of trajectories. If it could be achieved through a chaotic system, different initial conditions and a third component to control the system, that the system would exhibit different periodic and chaotic trajectories, we could tune the output through evolutionary algorithms to show a desired solution trajectory to simplify the emergence of locomotion. As it turns out, simple limiters, introduced in the next section, are a very natural way of influencing chaotic systems. Through their innate coupling with the chaotic system, because they express limitations in the state space in the same 'language' as the system and are hence elegant by simplicity, simple limiters could be the key to control the system into different trajectories for periodic and chaotic movement.

\subsection{Simple limiter control}
% rev. 1

Simple limiter control, first introduced in the paper by Ned J. Corron et al. in 2000 \cite{bib:Corron2000}, is much different from other preceeding chaos control strategies. A limiter expresses a limitation in the state space of the chaotic system and thereby suppresses certain periodic orbits. These periodic orbits can not be reached anymore by the state, the trajectories can never contain it. However, interesting is that through the limiter we can in fact stabilize the remaining, reachable periodic orbits. The seemingly unpredictable trajectory always deterministically switches always at the same point to the same periodic orbits in the phase space. Using the simple limiter, we prevent the switching at some specified point and leave the system state no other choice than to stay on its current orbit, and to repeat its trajectory, and hence is stabilized on a trajectory of a certain periodicity.
In a natural system such as the actuated leg joint, this limiter can be much more subliminal, be it the weight or dimensions of the limbs or the limits, damping or friction of the joints, or even the fact the two physical objects can not interpenetrate each other.

\end{document}