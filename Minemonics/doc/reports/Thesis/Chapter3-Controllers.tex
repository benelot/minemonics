\documentclass[main]{subfiles}

\begin{document}
% Chapter Template
\setcounter{chapter}{2}

% Controllers
%  Uncoupled sinusoidal oscillators
%  Chaotic systems
%  Chua Circuit
%   Simple limiter control in Mathematica
\chapter{Controllers} % Main chapter title

\label{Chapter\thechapter} % Change X to a consecutive number; for referencing this chapter elsewhere, use \ref{ChapterX}

\lhead{Chapter \thechapter. \emph{Controllers}} % Change X to a consecutive number; this is for the header on each page - perhaps a shortened title

\section{Adaptive controllers for [?]}

Starting in the industrial revolution, people have started to automatize production and thereby lowered the costs of production of almost any goods. Today's production robots can perform very sophisticated manipulation as is required for large-scale products such as cars of smaller scale products such as electronic chips. However, the best robots to date are robots very specifically built for the task at hand. Many of them totally lack autonomy or adaptability, even though the morphology could theoretically be reused for another task. But since the task does not require any learning, adaption or autonomous behavior, it is not built in. As more recent developments strive for more autonomy and adaptivity in robots, more adaptable controllers need to be developed. Tasks requiring autonomy and adaptivity are rescue, detection or exploration scenarios of robots performing outdoor. Such a task is generally very versatile, as it is totally unknown what situation has to be expected and what problems will have to be solved. This means that problems have to be solved in a much more robust manner than it has to be done in a supervised environment. Also the importance of low latency control arises because the system has to be reactive to its environment and is only in rare cases able to plans its solution very much ahead. The requirements for such a controller therefore are to be highly adaptive on one hand and with very low latency on the other hand. Many different approaches exist, [?], however most of them are highly engineered approaches, so they suffer from high power and latency because traditional problem solving generally includes proper planning.  When compared to nature, these approaches are far from competitive. An ant for example can leave its nest, go out into the unknown, find prey, communicate with other ants to collaboratively fight it, then take it apart and bring it back to the nest. This highly complex task requires []

Now that we understand the evolutionary optimization of the simulator, let us look into the controllers for the creatures a little bit more. 

\section{Uncoupled sinusoidal oscillators}

Before the actual adaptive controller was tested, a non-adaptive controller was used to check if the evolutionary optimization is able to come up with solutions of a high enough variety in order to find valuable creature controller solutions to make a creature locomote on a flat terrain. The controller is fully distributed within the body of the creature and runs completely uncoordinated. The controller consists of many subcontrollers, whereas one subcontroller controls one degree of freedom in a certain joint. The controller's output is a simple sinusoidal signal defined by the parameters of amplitude and frequency and the X and Y offset in signal space. The ranges are $\text{amplitude} \in [0,0.5]$, $\text{frequency} \in [0.1,4]$, $\text{offset}_X \in [0,2\Pi]$ and $\text{offset}_Y \in [0,1]$. So every time a Morphogene branch defining a joint is mutated, the sinusoidal subcontroller gets randomly initialized as well. The sinusoidal signal was then fed into a PID position controller to make the limb precisely follow the sinusoidal signal. Thereby the controller's solution is basically a set of sinusoids, that when applied to the coevolved morphology results in a locomotion movement that moves the creature forward. The controller type was tested in an experiment on the simulator and successful solutions were evolved

\section{Chaotic systems}

\lipsum[2]

\subsection{Chua Circuit}

According to [<ref name="Kennedy1993">{{cite journal
  | last = Kennedy
  | first = Michael Peter
  | title = Three steps to chaos - Part 1:Evolution
  | journal = IEEE Trans. on Circuits and Systems
  | volume = 40
  | issue = 10
  | pages = 640
  | publisher = Institute of Electrical and Electronic Engineers 
  | location = 
  | date = October 1993
  | url = http://www.eecs.berkeley.edu/~chua/papers/Kennedy93.pdf
  | issn = 
  | doi = 10.1109/81.246140
  | id = 
  | accessdate = February 6, 2014}}</ref>], for a circuit with time-varying output and no time-varying input that is built from electronic components such as resistors, capacitor and inductors, three criteria must hold in order to display chaotic behavior. It must contain:
  \begin{enumerate}
  \item at least one nonlinear element (where piecewise linear is sufficient)
  \item at least one locally active resistor
  \item at least three energy-storage elements
  \end{enumerate}
  
The simplest electronic circuit that satisfies these criteria is the Chua Circuit. Chua's circuit is chaotic system that can be built in the form of a simple electronic circuit. It was invented by Leon O. Chua when he visited the Waseda University in Japan in 1983. The circuit can be seen as a nonperiodic oscillator, that differently from an ordinary electronic oscillator never repeats its waveform. The circuit, remarkable because of its simplicity and rich variety of bifurcations and the presence of chaotic behavior, is one of the few physical systems for which the presence of chaos has been proven mathematically [again Kennedy1993].

\begin{figure}[!h]
\centering
\missingfigure[figwidth=1\textwidth]{Figure of the chua circuit and the piecewise linear resistor.}
\caption[The chua circuit]{The chua circuit}
\label{figure:chuacircuit}
\end{figure}

Using Kirchhoff's circuit laws to derive the equations of the chua circuit, we find the following nonlinear ordinary differential equations with the variables x(t), y(t) and z(t).

\begin{align*}
\frac{dx}{dt}&=\alpha [y-x-f(x)] &\frac{dx}{dt}\text{ is the voltage across the capacitor }C_1\\
RC_2\frac{dy}{dt}&= (x-y+Rz) &\frac{dy}{dt}\text{ is the voltage across the capacitor }C_2\\
\frac{dy}{dt}&=\beta (x-y+Rz) &\text{with } \frac{1}{RC_2} \text{ being }\beta\\
\frac{dz}{dt}&=-\gamma y &\frac{dz}{dt}\text{ is the current across the inductor }L_1\\
f (x) &= \frac{m_1 x + (m_0 - m_1)}{2 (| x + 1 | -| x - 1 |)} &f(x)\text{ describes the response of the piecewise linear resistor}
\end{align*}

\begin{comment}
dx/dt = c1*(y - x - f (x)) // m0 : slope in outer region
    dy/dt = c2*(x - y + z)    // m1 : slope in inner region
    dz/dt = -c3*y         // b : Breakpoints
    f (x) = m1*x + (m0 - m1)/2*(| x + 1 | -| x - 1 |)
\end{comment}

The chua circuit was chosen as a model system of a chaotic controller, because of its simplicity in the equations as well as its prior usage for simple limiter control in [Corron.], where Corron et al. showed on two different chaotic circuits, namely the double pendulum and the chua circuit, how appropriate simple limiters can be applied to each system so that the chaotic behavior of each unbounded system could be tuned to show behavior of different periodicity. Important to mention is that the Chua Circuit is not meant to be a model for appropriate leg movement, but the goal is to show that using simple limiter control to control an arbitrary chaotic system, the different periodicities generated by the system can exhibit periodic leg movement to form different gaits and that the change of the simple limiters can lead to a change in periodic movement, thereby adaption to different environments could occur. 

\subsubsection{Simple limiter control in Mathematica}

Before the circuit is used as a chaotic controller, the Chua circuit was modelled in Mathematica to show its original chaotic behavior and how it can be influenced by simple limiters to exhibit different periodicities. The figure \ref{figure:chaoticchuacircuit} shows the Chua circuit's multiscroll attractor when using initial conditions \([-1.5,0,0]\) without any limiter.

\begin{figure}[!h]
\centering
\missingfigure[figwidth=1\textwidth]{Figure of the multiscroll attractor}
\caption[The multiscroll attractor]{The multiscroll attractor}
\label{figure:chaoticchuacircuit}
\end{figure}

When we introduce a simple soft limiter \[softLim(x) = \frac{1}{2} \left(\tanh\left(\frac{limitValue - x}{softness}\right) + 1\right)\] with the limiter response shown in figure \ref{figure:softlimiterresponse} as follows:

\begin{align*}
\frac{dx}{dt}&=\alpha [y-x-f(x)] \\
\frac{dy}{dt}&=\beta (x-y+Rz)\\
\frac{dz}{dt}&=-\gamma y softLim(x)\\
f (x) &= \frac{m_1 x + (m_0 - m_1)}{2 (| x + 1 | -| x - 1 |)}
\end{align*}

\begin{figure}[!h]
\centering
\missingfigure[figwidth=1\textwidth]{Figure of soft limiter response}
\caption[Figure of soft limiter response]{Figure of soft limiter response}
\label{figure:softlimiterresponse}
\end{figure}

Intially a hard limiter defined as a piecewise function was chosen, however a soft limiter is a better model for physical limiters, because in physics, no hard limiter exists. Using the $softness$ parameter of the limiter, the softness of the limiter can be tuned where a \(softness = 10\) means very soft and a \(softness = 0.1\) means a very hard limiter, the $limitValue$ parameter defines the threshold value at which the limiter starts to be applied. We set the softness to a values of \(softness=0.13\) to approximate a rather hard limiter. With the above mentioned initial conditions, the trajectory in state space stays within the bounding box of \(x \in [-2.8717,2.2417], y \in [-0.380467,0.386978],z \in [-3.59156,3.62959]\).
If we now change the $limitValue$ parameter from \(>3.19\) which corresponds to no limiter, to a value within \(limitValue \in [2.4,3.19[\), we can observe different chaotic trajectories. The limiter influences the trajectory by limiting the current across the inductor \(L_1\) where as the limitation depends on how much the x variable of the state approaches the limiter.

\begin{figure}[!h]
\centering
\missingfigure[figwidth=1\textwidth]{Figure of chaotic behaviors in range 2.4-3.19}
\caption[Figure of chaotic behaviors in range 2.4-3.19]{Figure of chaotic behaviors in range 2.4-3.19}
\label{figure:chaotictrajectories}
\end{figure}

When choosing a \(limitValue \in [2.28,2.4[\) one can observe that the limiter more and more inhibits chaotic behavior and reduces the number of times the trajectory switches back into the uppermost scroll of the multiscroll. 

\begin{figure}[!h]
\centering
\missingfigure[figwidth=1\textwidth]{Figure of behaviors in range 2.28-2.4}
\caption[Figure of behaviors in range 2.28-2.4]{Figure of behaviors in range 2.28-2.4}
\label{figure:chaotictrajectories}
\end{figure}

With a value of \(limitValue \in [2,183, 2.28]\) the state only stays in the lowermost scroll and shows periodic behavior. With a value of \(limitValue \in [2,183, 2.28]\) the state only stays in the lowermost scroll and shows periodic behavior. The table \ref{table:periodicities} shows different limit values for different periodicities.

\begin{comment}
\begin{table}
\center
\begin{tabular}{|l|l|l|l|l|l|l|l|l|l|l|l|l|}
   \hline
   limitValue & 2.215 & 2.211 & 2.2105 & 2.2101 & 2.2101 &  2.208 & 2.207 & 2.206 & 2.205 & 2.1995 & 2.183 \\
   \hline
   Limit Cycle Period & 24 & 11 & 9 & 8 & 7 & 6 & 5 & 4 & 3 & 2 \\
   \hline
\end{tabular}
\caption{Whatever}
\label{table:periodicities}
\end{table}
\end{comment}

\begin{table}
\renewcommand{\arraystretch}{1.2}
\center
\begin{tabular}{@{}ll@{}}
	\toprule
   \(limitValue\) & Limit Cycle\\
   \midrule
   2.215 & Period 24 \\ 
   2.211 & Period 11 \\
   2.2105 & Period 9 \\
   2.2101 & Period 8 \\
   2.208 & Period 7 \\
   2.207 & Period 6 \\
   2.206 & Period 5 \\
   2.205 & Period 4 \\
   2.1995 & Period 3 \\
   2.183 & Period 2 \\
   \bottomrule
\end{tabular}
\caption{Different limit values resulting in trajectories of different periodicity}
\label{table:periodicities}
\end{table}

From the table \ref{table:periodicities} it gets visible how densely packed the different periodicities are in the limiter space. A sensitivity to the simple limiter reveals itself. But it gets more interesting as we move the limiter even further towards 0. Instead of showing a period 1 as one might expect, the periodicity becomes higher again for \(limitValue \in [2.183,2.15]\) and lowers again for \(limitValue \in [2.15,1.83]\). A self-similar pattern of increasing and decreasing periodicities can be observed.

\todo[inline]{Improve simple limiter control in Mathematica}

\end{document}