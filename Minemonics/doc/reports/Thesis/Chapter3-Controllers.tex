\documentclass[main]{subfiles}

\begin{document}
% Chapter Template
\setcounter{chapter}{2}

% Controllers
%  Uncoupled sinusoidal oscillators
%  Chaotic systems
%  Chua Circuit
%   Simple limiter control in Mathematica
\chapter{Controllers} % Main chapter title

\label{Chapter\thechapter} % Change X to a consecutive number; for referencing this chapter elsewhere, use \ref{ChapterX}

\lhead{Chapter \thechapter. \emph{Controllers}} % Change X to a consecutive number; this is for the header on each page - perhaps a shortened title

\section{Adaptive controllers for [?]}

Starting in the industrial revolution, people have started to automatize production and thereby lowered the costs of production of almost any goods. Today's production robots can perform very sophisticated manipulation as is required for large-scale products such as cars of smaller scale products such as electronic chips. However, the best robots to date are robots very specifically built for the task at hand. Many of them totally lack autonomy or adaptability, even though the morphology could theoretically be reused for another task. But since the task does not require any learning, adaption or autonomous behavior, it is not built in. As more recent developments strive for more autonomy and adaptivity in robots, more adaptable controllers need to be developed. Tasks requiring autonomy and adaptivity are rescue, detection or exploration scenarios of robots performing outdoor. Such a task is generally very versatile, as it is totally unknown what situation has to be expected and what problems will have to be solved. This means that problems have to be solved in a much more robust manner than it has to be done in a supervised environment. Also the importance of low latency control arises because the system has to be reactive to its environment and is only in rare cases able to plans its solution very much ahead. The requirements for such a controller therefore are to be highly adaptive on one hand and with very low latency on the other hand. Many different approaches exist, [?], however most of them are highly engineered approaches, so they suffer from high power and latency because traditional problem solving generally includes proper planning.  When compared to nature, these approaches are far from competitive. An ant for example can leave its nest, go out into the unknown, find prey, communicate with other ants to collaboratively fight it, then take it apart and bring it back to the nest. This highly complex task requires []

Now that we understand the evolutionary optimization of the simulator, let us look into the controllers for the creatures a little bit more. 

\section{Uncoupled sinusoidal oscillators}

Before the actual adaptive controller was tested, a non-adaptive controller was used to check if the evolutionary optimization is able to come up with solutions of a high enough variety in order to find valuable creature controller solutions to make a creature locomote on a flat terrain. The controller is fully distributed within the body of the creature and runs completely uncoordinated. The controller consists of many subcontrollers, whereas one subcontroller controls one degree of freedom in a certain joint. The controller's output is a simple sinusoidal signal defined by the parameters of amplitude and frequency and the X and Y offset in signal space. The ranges are $\text{amplitude} \in [0,0.5]$, $\text{frequency} \in [0.1,4]$, $\text{offset}_X \in [0,2\Pi]$ and $\text{offset}_Y \in [0,1]$. So every time a Morphogene branch defining a joint is mutated, the sinusoidal subcontroller gets randomly initialized as well. The sinusoidal signal was then fed into a PID position controller to make the limb precisely follow the sinusoidal signal. Thereby the controller's solution is basically a set of sinusoids, that when applied to the coevolved morphology results in a locomotion movement that moves the creature forward. The controller type was tested in an experiment on the simulator and successful solutions were evolved

\section{Chaotic systems}

\lipsum[2]

\subsection{Chua Circuit}

\lipsum[3]

\subsubsection{Simple limiter control in Mathematica}

\lipsum[4]

\todo[inline]{Give the Chapter 3 some meaningful content.}

\end{document}