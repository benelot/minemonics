% Chapter 6
\documentclass[main]{subfiles}
\setcounter{chapter}{5}

\begin{document}
\chapter{Conclusion} % Main chapter title

\label{Chapter6} % For referencing the chapter elsewhere, use \ref{Chapter1} 

\lhead{Chapter 6. \emph{Conclusion}} % This is for the header on each page - perhaps a shortened title

%----------------------------------------------------------------------------------------
%Conclusion

In this thesis, the creation of a prototype of a pneumatic system for hand prosthesis has been shown. It already shows features like force output and speed that is comparable to or even higher than other prosthetic systems. There is definitely space for optimization such as the size of the tank and its maximum pressure. Also the controller is not yet optimal as it lacks force control as well as hand grasping pattern finger coordination. But it is important to see the result in its time relations. A system like this is an early prototype that has been created in just 6 months which is a very short time for the development of such an actuation system. Also the investment of around 1000\$ is very low for prototyping when compared to the investments that are done to develop real commercial prostheses. Working on a very low funding as it is often the case in research still gives the designer the opportunity to work out the best from the bought components. It also forces the designer to do proper research when it comes to finding components for the system that perfectly match or reach the best performance compared with the optimal requirements. Furthermore, during the search for appropriate products it got obvious that pneumatic systems mostly are built bulky because they are normally immobile and therefore do not have constraints such as weight or size. If a system needs high precision but the user should be able to work with a small tool such as several systems for surgery used, the valves are never located in the tool itself but are placed somewhere in the back where big and bulky valves can be used. The created system moreover is not able to nicely exploit the fact that air is highly compressible. This is due to the compressor not being able to compress the air up to higher pressures than 5 bar. Otherwise, a lot of air could have been compressed into a tank and the prosthesis could have worked without refilling the tank for a longer time. A tank could be then be used which withstands higher pressures, so that with the aid of a pressure regulator the pressure could be regulated for the valves and provide constant air pressure for a longer time.\\

Apart from the problems that are still open to be solved, several achievements can be mentioned.

\section{Force of the hand}
With the current setting, the hand is able to hook 13 kilograms of weight with a finger with a triarticular muscle. For a finger with two biarticular muscles, that would mean that it could hook a weight of 26 kg. For a common hook grasp involving four fingers, the muscles could provide a force to hold about 104 kg. Compared to other prostheses, this is a very strong force that can be controlled precisely to lift weights of grasp objects in a manner that is definitely valid for most common ADLs.

\section{Speed of the hand}
From the experiment in Chapter \ref{Chapter5}, it is shown that bottlenecks of the speed of pneumatic artificial muscles is mostly limited by the flow of the valve. In our case, that means that the hand can fully open or close at a frequency of 1 Hz. Compared to other systems, this is a pretty normal value. Using valves with a higher flow shows that the speed of the hand can easily be improved so that the hand reaches an opening and closing frequency of 16 Hz.

\section{Precision of position of the hand}
The precision of position control of the hand shows that the system reaches the desired position in less than one second and has very little overshoot over the position. Since the hand is compliant, even with some overshoot the hand will basically stop when touching an object and will start to provide force on the object to hold it.

\section{Future work}

Since the prototype of the Aeromanus shows interesting features that could be developed further, it is time to outline the possibilities of future work on the device. 

\subsection{Creation of the complete hand}

The first project to do would be to finish the whole hand to create a scientific platform from it. The device is very promising and it would definitely help to get further results on the control of pneumatic muscles. 

\subsection{Integration of a hierachical control structure}

Since the hand is already very bio-inspired by mimicking human muscles in terms of compliance and strength, the Aeromanus hand as a full hand could be interesting to work with a bio-inspired hierarchical control structure. 


\subsection{Aesthetics of a prosthesis}

Currently the actual hand bones of the Aeromanus hand are only the skeletal structure of a hand similar to the bones of the human hand. The morphology of a real human hand contains much more levels of compliance that could serve the strength and stability of the grasp. Currently only the muscles of the prosthesis are adding compliance to the grasp. In a natural hand, also the tissue of the fingers and palm are working as cushions and therefore fit the shape of an object by wrapping into its structure \cite{Kargov2004}. This improves the contact area and is shown to stabilize the grip \cite{Kargov2004}.

%----------------------------------------------------------------------------------------

\subsection{Sensor input for improve sensory-motor coupling}

The device currently lacks sensors. The human hand contains 14 different types of sensors to give the human a proper model of what is felt when touching or what is held when grasping and carrying. Sensors in a prosthesis can give a feedback to the user on one side so that he can adjust its control to get a control over its hand manipulation. On the other hand it can help to lower the necessary inputs of the amputee since the hand can process the sensory input by itself.

%----------------------------------------------------------------------------------------

\subsection{Control of grip pattern with a neuronal headset}
In the current version of the Aeromanus hand, there is just a simple type of control included. At the moment, the muscles of the system can be inflated and deflated by the use of a potentiometer. Classical control of the prosthesis would be via EMG sensors to sense the muscle actuation in the amputee's residual limb. After some classification of the signal and an appropriate control strategy, the controller chooses a certain type of actuation such as a grasping pattern and the amputee can then control the strength of the grasp. A very interesting control strategy would be to control the grasp pattern via a neuronal headset that is placed on the head of the user. This would not only improve the size of the human-computer interface to control the high dexterity that a prosthesis can achieve, but also would give a more intuitive way of controlling it.


\subsection{Alternatives for intrinsic muscles}
Even though it was planned initially to include intrinsic muscles into the Aeromanus hand, it was very difficult to create very thin McKibben muscles that would properly work of $\leq$ 4 bar and have a contraction ratio high enough to actuate the fingers. One advantage of the intrinsic muscles would be that they do not have to be strong as they are mostly used for precise movements and not for actions involving heavy loads. As the requirements are mentioned, it should get clear that it could be a possibility to use alternatives to pneumatic actuators to get an equivalent to intrinsic muscles. Very small servomotors could be used to move the fingers for the precise actions normally taken over by the intrinsic muscles and pneumatic artificial muscles could be used to provide the necessary force of a power grip or a stronger tripod finger pinch. This option could prove itself to be worth a try in a further going project.

\end{document}