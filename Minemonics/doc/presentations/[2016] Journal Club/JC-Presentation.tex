\documentclass{beamer}
% use this instead for 16:9 aspect ratio:
%\documentclass[aspectratio=169]{beamer}
\usepackage{etex}
\usepackage{verbatim}
\reserveinserts{28}
\usetheme{ETHbeamer}

\colorlet{ETHcolor1}{ETHc}
\colorlet{ETHcolor2}{ETHc}

\author{Benjamin Ellenberger}
\institute{INI:  Institute of Neuroinformatics}

\title{Multiple chaotic central pattern generators with learning for legged locomotion and malfunction compensation}

\date{2016-01-27}

%%TODO: Optimize with presentation tips

% uncomment if you do not want to use a department logo
%\deplogofalse

\usepackage{color}
\usepackage{listings}
\usepackage{caption}
\usepackage{todonotes}

\DeclareCaptionLabelFormat{algocaption}{Algorithm} % defines a new caption label as Algorithm x.y

\lstnewenvironment{algorithm}[1][] %defines the algorithm listing environment
{   
    \lstset{ %this is the stype
        numbers=left, 
        numberstyle=\tiny,
        basicstyle=\tiny, 
        keywordstyle=\color{black}\bfseries\em,
        keywords={, initialize, if, then, else, foreach, do, while, begin, end, repeat, until, } %add the keywords you want, or load a language as Rubens explains in his comment above.
        numbers=left,
        xleftmargin=.04\textwidth,
        #1 % this is to add specific settings to an usage of this environment (for instnce, the caption and referable label)
    }
}
{}

\begin{document}

\titleframe

\section{}

\section{Introduction to the basic mechanism}

\begin{comment}
-Many researchers have proved that CPG-based control is an effective method for locomotion control of bio-inspired walking robots.

-A chaotic system can be controlled into showing periodic dynamics, so as to be implemented as a CPG to accomplish the locomotion control of a bio-inspired hexapod robot.

-So far, the chaotic CPG system contains only one oscillator, which has difficulties dealing with leg malfunction.

-We extend the single chaotic CPG system to multiple coupled CPG systems. Therefore each leg can perform identical patterns for normal walking or independent patterns to cope with malfunction.



-In the end, we show the real robot experiments to verify the performance of our methods.

Methods
-The single chaotic CPG is extended to multiple CPGs, specifically to six for a hexapod robot and to four for a quadruped robot.
-We choose a master CPG arbitrarily which dominates the pattern when synchronized, while the other CPGs are clients which can oscillate independently when desynchronized.
-When some legs are malfunctioning, they lose synchronization and can show different patterns such that the robot can maintain its forward walking in a straight line.
-An annealing-based learning algorithm is applied to automatically find the best matching oscillation frequencies to compensate for the malfunction.

Results
-The proposed algorithm was evaluated first in a physics simulation of a quadruped as well as a hexapod robot.
-It was also tested on the real hexapod robot AMOSII with three scenarios where one, two and three legs were disabled.
-In these three experiments, the robot deviated from its original trajectories and failed to pass the tunnel when the remaining legs with identical periods.
-The robot can walk in a straight line and successfully pass the tunnel when the learned combination of leg periods was employed.

Conclusions
-We demonstrate that multiple coupled chaotic CPGs with learning can be used for legged locomotion and malfunction compensation.
-When all legs are function, al the CPG units stay synchronized to show identical movement; if some legs are malfunctioning, the CPGs can desynchronize and learn a proper combination of leg periods to compensate for the malfunctioning legs.
\end{comment}


% Are we alone in the universe?
\begin{minimalframe}
    \hspace*{-1.1cm}
    \vspace*{3cm}
    \includegraphics[width=1.2\textwidth,clip]{figs/creatures/Minemonics-05112015_190947481.jpg}
    
\end{minimalframe}

\begin{frame}
\frametitle{Normal Frame}
\(x_i(t+1) = \sigma\left(\theta_i + \sum\limits^2_{j=1}w_{ij} x_j(t) + c_i^{(p)(t)}\right) \, for~i \in \{1,2\}\)
\todo[inline]{Give the Chapter 7 some meaningful content.}
\end{frame}

\begin{frame}
\frametitle{Normal Frame}
\(c_i^{(p)} = \mu^{(p)}(t)\sum\limits^{2}_{j=1}w_{ij}\Delta_j(t)\)
\end{frame}

\begin{frame}
\frametitle{Normal Frame}
\(\Delta_j(t) = x_j(t) = x_j(t-p)\)
\end{frame}

\begin{frame}
\frametitle{Normal Frame}
\(\mu^{(p)} = \mu^{(p)}(t) + \lambda\frac{\Delta^2_1(t) + \Delta^2_2(t)}{p} \, with~\lambda=0.05\)
\end{frame}


\begin{frame}[fragile]
\frametitle{Normal Frame}
\begin{algorithm}[mathescape]
initialize $C(1)= [1/4 1/24; 4; 4; 4; 4; 4]; \Delta\phi= 0.0; E_1 = 0.0$
repeat:
    At trials n
    do
        randomly pick a leg l, $l \in [R1; R2; R3; L1; L2; L3]$
        change the period of leg l to a random value, $P(l) \in [4; 5; 6; 8; 9]$
        compare this combination of leg period, $C'(n)$, to the walking records
    until $C'(n)$ is a new combination of leg periods
    
    run the robot
    // calculate the evaluation function and its variation
    $E_n = \Delta\phi$
    $\Delta E = E_n - E_{n-1}$
    
    decide the combination of leg periods
    
    if $\Delta E < 0$ then
        $C(n) = C'(n)$
    else
        if  $x \geq e^{\beta \Delta E}$ then
            $C(n) = C'(n)$
        else
            $C(n) = C(n - 1)$
        end if
    end if
until: The evaluation function $E_n$ is less than a required value $E_{req}$
\end{algorithm}
\end{frame}

\begin{frame}
\newcommand{\quadrat}{(0,0mm)--(0mm,5mm)--(5mm,5mm)--(5mm,0mm)--(0mm,0mm);}
\begin{center}
    \hspace{-8mm}
    \begin{tikzpicture}[overlay]
        {\draw[ETHa,fill=ETHa] \quadrat}\label{ETH1}
    \end{tikzpicture}
    \hspace{10mm}
    \begin{tikzpicture}[overlay]
        {\draw[ETHb,fill=ETHb]\quadrat}\label{ETH2}
    \end{tikzpicture}
    \hspace{10mm}
    \begin{tikzpicture}[overlay]
        {\draw[ETHc,fill=ETHc]\quadrat}\label{ETH3}
    \end{tikzpicture}
    \hspace{10mm}
    \begin{tikzpicture}[overlay]
        {\draw[ETHd,fill=ETHd] \quadrat}\label{ETH4}
    \end{tikzpicture}
    \hspace{10mm}
    \begin{tikzpicture}[overlay]
        {\draw[ETHe,fill=ETHe] \quadrat}\label{ETH5}
    \end{tikzpicture}
    \hspace{10mm}
    \begin{tikzpicture}[overlay]
        {\draw[ETHf,fill=ETHf] \quadrat}\label{ETH6}
    \end{tikzpicture}
    \hspace{10mm}
    \begin{tikzpicture}[overlay]
        {\draw[ETHg,fill=ETHg] \quadrat}\label{ETH7}
    \end{tikzpicture}
    \hspace{10mm}
    \begin{tikzpicture}[overlay]
        {\draw[ETHh,fill=ETHh] \quadrat}\label{ETH8}
    \end{tikzpicture}
    \hspace{10mm}
    \begin{tikzpicture}[overlay]
        {\draw[ETHi,fill=ETHi] \quadrat}\label{ETH9}
    \end{tikzpicture}\\
    \vspace*{1em}
    {\Huge \textbf{Discussion!}}
\end{center}
\vspace*{1.5em}

\begin{itemize}
\item Any questions?
\item What experiments do you have in mind?
\item What else would you change, extend, enhance, improve etc.?
\item If you have any ideas later, email me: be.ellenberger@gmail.com
\item You can look at my progress: https://github.com/benelot/minemonics
\end{itemize}

\end{frame}


\frame{

  \frametitle{References}
  \begin{itemize}
  \item Sims K. - Evolving Virtual Creatures (1994)
  \item Sims K. - Evolving 3D Morphology and Behavior by Competition (1994)
  \item Krcah P. - Evolving Virtual Creatures Revisited (2007)
  \item Corron N. et al. - Controlling Chaos with Simple Limiters (2000)
  \item Schmidt N. - Bootstrapping perception using information theory: case studies in a quadruped running robot running on different grounds (2013)
  \item Stoop R. - Theory and Simulation of Neural Networks (2014)
  \end{itemize}
}
\note{}

\begin{comment}

% Example slides
\begin{frame}
\textbf{Some mathematical specialities}

\ETHbox{0.8\textwidth}{% define the ETHbox
  \begin{theorem}[Murphy (1949)]\label{murphy}
    Anything that can go wrong, will go wrong.
  \end{theorem}
}

\begin{proof}
  A special case of Theorem \ref{murphy} is proven in %\citet{matthews1995}.
\end{proof}
\end{frame}

\begin{titlestyleframe}
\frametitle{Title Page}

\color{white} The title page is created using the \texttt{\textbackslash titleframe} command.

The title page background can also be used on other frames (or for a customised title frame) using the \texttt{titlestyleframe} environment.
\end{titlestyleframe}

\begin{frame}
\frametitle{Normal Frame}
The normal frame looks like this. It is created using the \texttt{frame} environment.
\end{frame}
\begin{inverseframe}
  \frametitle{Inverse Slides}
  %\color{white}
The inverted frame looks like this. It is created using the \texttt{inverseframe} environment.

 
\end{inverseframe}

\begin{minimalframe}
  \frametitle{Minimal Frame}
The minimal frame looks like this. It is created using the \texttt{minimalframe} environment.
  
\end{minimalframe}

\begin{minimalframe} 
\end{minimalframe}

\end{comment}

\end{document}
