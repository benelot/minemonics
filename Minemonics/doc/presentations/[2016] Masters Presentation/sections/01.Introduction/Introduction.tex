\providecommand{\main}{../..} %Important: It needs to be defined before the documentclass
\documentclass[../../main]{subfiles}

\begin{document}

\begin{frame}
\frametitle{Introduction}
I introduce things.
\begin{figure}[H]
\centering
\includegraphics[width=0.8\textwidth]{controllers/chua-circuit/Unlimited-chua-circuit.png}
\caption[The multiscroll attractor in the Chua circuit]{The multiscroll attractor generated by the Chua circuit without any simple limiter applied.}
\label{figure:chaoticchuacircuit}
\end{figure}
\end{frame}

\begin{frame}
\frametitle{Introduction II}
I introduce things.
\begin{figure}[H]
\centering
\includegraphics[width=0.2\textwidth]{controllers/chua-controller.pdf}

\caption[The chaotic chua controller]{The specification of the chaotic chua controller, its internal state and output signal. The internal state of the chaotic chua controller is three dimensional because the Chua circuit's equations are defined using three dimensions. The output signal is chosen to be the z dimension of the internal state and is therefore only one dimensional. The controller state and output are shown for the unlimited condition and an example of limited condition. The example limitation leads to a period two limit cycle. If the chaotic chua circuit controller is mutated during the mutation step of the simulator, the parameters are chosen from a uniform distribution out of the respective intervals.}
\label{figure:chua-controller}
\end{figure}
\end{frame}

\begin{comment}
\begin{frame}

  \frametitle{Contents}
  \tableofcontents[currentsection]
\end{frame}


\section{Something}

\frame{

  \frametitle{Evolving Virtual Creatures}
  
  \begin{columns}
   \column{0.3\textwidth}
 \begin{itemize}
	     \item a
    	\item b
    	\item c
     \end{itemize}
     
     \column{0.6\textwidth}
      %\includegraphics[width=2in, clip] {figs/creatures.jpg} 
  \end{columns}
  }
\note{}

\begin{frame}
\textbf{New colors and their names\\}
Here we show the different colors you can use. From left to right, this are the colors ETH1, ETH2, \ldots , ETH9.

\newcommand{\quadrat}{(0,0mm)--(0mm,5mm)--(5mm,5mm)--(5mm,0mm)--(0mm,0mm);}
\begin{center}
	\hspace{-8mm}
	\begin{tikzpicture}[overlay]
		{\draw[ETHa,fill=ETHa] \quadrat}\label{ETH1}
	\end{tikzpicture}
	\hspace{10mm}
	\begin{tikzpicture}[overlay]
		{\draw[ETHb,fill=ETHb]\quadrat}\label{ETH2}
	\end{tikzpicture}
	\hspace{10mm}
	\begin{tikzpicture}[overlay]
		{\draw[ETHc,fill=ETHc]\quadrat}\label{ETH3}
	\end{tikzpicture}
	\hspace{10mm}
	\begin{tikzpicture}[overlay]
		{\draw[ETHd,fill=ETHd] \quadrat}\label{ETH4}
	\end{tikzpicture}
	\hspace{10mm}
	\begin{tikzpicture}[overlay]
		{\draw[ETHe,fill=ETHe] \quadrat}\label{ETH5}
	\end{tikzpicture}
	\hspace{10mm}
	\begin{tikzpicture}[overlay]
		{\draw[ETHf,fill=ETHf] \quadrat}\label{ETH6}
	\end{tikzpicture}
	\hspace{10mm}
	\begin{tikzpicture}[overlay]
		{\draw[ETHg,fill=ETHg] \quadrat}\label{ETH7}
	\end{tikzpicture}
	\hspace{10mm}
	\begin{tikzpicture}[overlay]
		{\draw[ETHh,fill=ETHh] \quadrat}\label{ETH8}
	\end{tikzpicture}
	\hspace{10mm}
	\begin{tikzpicture}[overlay]
		{\draw[ETHi,fill=ETHi] \quadrat}\label{ETH9}
	\end{tikzpicture}
\end{center}

Please use no more then two of them in your presentation. The first two (counted from the left hand side) are reserved for the administration chapter of the ETHZ (first one for external presentation, second for internal), all others you can freely choose.

\end{frame}

\begin{frame}
\textbf{Some mathematical specialities}

\ETHbox{0.8\textwidth}{% define the ETHbox
  \begin{theorem}[Murphy (1949)]\label{murphy}
    Anything that can go wrong, will go wrong.
  \end{theorem}
}

\begin{proof}
  A special case of Theorem \ref{murphy} is proven in %\citet{matthews1995}.
\end{proof}
\end{frame}

\begin{titlestyleframe}
\frametitle{Title Page}

\color{white} The title page is created using the \texttt{\textbackslash titleframe} command.

The title page background can also be used on other frames (or for a customised title frame) using the \texttt{titlestyleframe} environment.
\end{titlestyleframe}

\begin{frame}
\frametitle{Normal Frame}
The normal frame looks like this. It is created using the \texttt{frame} environment.
\end{frame}

\begin{inverseframe}
  \frametitle{Inverse Slides}
  %\color{white}
The inverted frame looks like this. It is created using the \texttt{inverseframe} environment.
\end{inverseframe}

\begin{minimalframe}
  \frametitle{Minimal Frame}
The minimal frame looks like this. It is created using the \texttt{minimalframe} environment.
  
\end{minimalframe}

\end{comment}

\end{document}